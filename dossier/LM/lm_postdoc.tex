\documentclass[12pt]{lettre}



\usepackage[utf8]{inputenc}
\usepackage[T1]{fontenc}
\usepackage{lmodern}
\usepackage{eurosym}
\usepackage[frenchb]{babel}
\usepackage{numprint}

%\makeatletter
%\def\@texttop{}
%\makeatother

\begin{document}

\begin{letter}{Université Côte d'Azur
	
	Centre de référence \og Défis du Numérique \fg	
}
	
	\name{Khaoula BOUKIR}
	\address{Khaoula BOUKIR\\9 rue du Fresche Blanc\\44300 Nantes}
	\lieu{Nantes}
	\telephone{07 81 83 01 98}
	\email{khaoula.boukir@ls2n.fr}
	\nofax
	
	
	
	\def\concname{Objet :~} % On définit ici la commande 'objet'
	\conc{Lettre de motivation pour candidature à un poste de post-doctorant}
	\opening{Madame, Monsieur,}
	
	J'ai le plaisir de vous présenter, par cette lettre, ma candidature pour le poste de post-doctorant que vous proposez. Je suis actuellement doctorante en fin de thèse à l'Université de Nantes, rattachée à l'équipe STR du laboratoire LS2N\footnote{www.ls2n.fr/equipe/str/}.
	
	Diplômée d'un master 2 de recherche en "Automatique, Robotique et Informatique Appliquée" parcours "Temps Réel et Systèmes Embarqués", je mène actuellement mes travaux en thèse sur le sujet "Mise en œuvre de politiques d'ordonnancement temps réel multiprocesseur prouvée", dont la soutenance est prévue pour la fin de l'année 2020. J'ai eu l'occasion durant mon expérience en recherche d'étudier les notions liées à l'utilisation des méthodes formelles dans la vérification des systèmes d'exploitation temps réel, avec un focus sur la vérification par model-checking des implémentation d'ordonnanceur temps réel. Mon CV, en pièce jointe, dresse un bilan des travaux et articles que j'ai pu mener et publier au cours de ma thèse ainsi que mes parcours universitaire et professionnel.
	
	
	%J'ai également eu l'occasion d'assurer des missions d'enseignement	par le biais de TD/TP qu'on m'a confié à l'\'{E}cole Centrale de Nantes (2016/2017) et l'IUT de Nantes au sein de votre département (2017/2018, 2018/2019). Mes activités d'enseignement sont résolument tournées vers l'informatique et les systèmes électroniques. 
	
	
	Souhaitant mener par la suite une carrière d'enseignant chercheur, le poste que vous offrez, sera une opportunité pour perfectionner mes compétences en matière de recherche. Par ailleurs, ma formation initiale combinée avec mes expériences dans la thématique des méthodes formelles constituent des atouts majeurs qui permettront de mener à bien les missions qui me seraient destinées au sein de votre unité de recherche.

	Convaincue qu’une lettre ne peut que partiellement rendre compte de ma motivation, je reste à votre
	entière disposition pour un éventuel entretien afin de vous convaincre de ma motivation.
	
	\closing{Je vous prie du soin particulier que vous voudrez bien apporter à ma candidature et vous prie de croire, Madame, Monsieur, à l’assurance de ma considération.}
	
\end{letter}


\end{document}  